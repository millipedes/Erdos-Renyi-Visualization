\documentclass{report}

\usepackage{algorithm}
\usepackage{algpseudocode}
\usepackage{amsmath}
\usepackage{amsfonts}
\usepackage{amssymb}
\usepackage{enumitem}
\usepackage{fancyhdr}
\usepackage{graphicx}
\usepackage{mathrsfs}
\usepackage{sectsty}
\usepackage{soul}
\usepackage{titlesec}

\newcommand{\n}{$\char`\\ n$}
\newcommand{\tab}{$\quad$}
\newcommand{\smalls}{\par\smallskip}
\newcommand{\meds}{\par\medskip}
\newcommand{\biigs}{\par\bigskip}
\newcommand{\HRule}{\rule{\textwidth}{1pt}}
\newcommand{\csfont}[1]{\fontfamily{cmtt}\selectfont #1}
\newcommand{\XOR}{\mathbin{\char`\^}}
\newcommand{\smalltitle}[1]{\textit{#1}}
\newcommand{\consection}[1]{\addcontentsline{toc}{section}{#1}\section*{#1}}
\newcommand{\consubsection}[1]{\addcontentsline{toc}{subsection}{#1}\subsection*{#1}}
\newcommand{\consubsubsection}[1]{\addcontentsline{toc}{subsubsection}{#1}\subsubsection*{#1}}
\newcommand{\eren}{Erd\H{o}s-R\'enyi}

\parindent 0pt

\pagestyle{fancy}
\fancyhf{}
\rhead{\thepage}
\lhead{\textit{Millipeds Git Repo: Program Explanation}}
\rfoot{}

\begin{document}
\begin{titlepage}
    \begin{center}
        \vspace{0.25cm}
        \hrule
        \vspace{0.1cm}
        \hrule
        \vspace{0.25cm}
        {\LARGE{\textbf{Millipeds Git Repo}}}
        \par
        \vspace{0.25cm}
        {\large{\textbf{Program Explanation}}}
        \par
        \vspace{0.25cm}
        {\large{\textbf{Date \today}}}
        \vspace{0.25cm}
        \hrule
        \vspace{0.1cm}
        \hrule
        \vfill
    \end{center}
\end{titlepage}

\tableofcontents
\par
\newpage

\consection{Pixel}
  A pixel data structure in this program is defined as follows:
  \biigs
  \begin{tabular}{cl}
    \hline
    \csfont{1} & \csfont{typedef struct PIXEL$\_$T $\{$} \\
    \csfont{2} & \tab \csfont{int r;}                    \\
    \csfont{3} & \tab \csfont{int g;}                    \\
    \csfont{4} & \tab \csfont{int b;}                    \\
    \csfont{5} & \csfont{$\}$ * pixel;}                  \\
    \hline
  \end{tabular}
  \biigs
  And can be understood as a collection:
  $$\text{Pixel} = (r, g, b)$$
  Such that r, g, and b are signed integers.
\consection{Canvas}
  A canvas data structure in this program is defined as:
  \biigs
  \begin{tabular}{cl}
    \hline
    \csfont{1} & \csfont{typedef struct CANVAS$\_$T $\{$} \\
    \csfont{2} & \tab \csfont{pixel ** values;}           \\
    \csfont{3} & \tab \csfont{int height;}                \\
    \csfont{4} & \tab \csfont{int width;}                 \\
    \csfont{5} & \csfont{$\}$ * canvas;}                  \\
    \hline
  \end{tabular}
  \biigs
  From this definition we can think of a canvas as a matrix:
  $$\text{Canvas} = \begin{bmatrix}
                      \text{Pixel}_{11}             & \dots & \text{Pixel}_{1\text{Width}}             \\
                      \text{Pixel}_{21}             & \dots & \text{Pixel}_{2\text{Width}}             \\
                      \dots                         & \dots & \dots                                    \\
                      \text{Pixel}_{\text{Height}1} & \dots & \text{Pixel}_{\text{Height}\text{Width}} \\
                    \end{bmatrix}$$
\consection{Node}
  A Node data structure in this program is defined as:
  \biigs
  \begin{tabular}{cl}
    \hline
    \csfont{1} & \csfont{typedef struct NODE$\_$T $\{$} \\
    \csfont{2} & \tab \csfont{char * name;}             \\
    \csfont{3} & \tab \csfont{pixel color;}             \\
    \csfont{4} & \tab \csfont{int fx;}                  \\
    \csfont{5} & \tab \csfont{int fy;}                  \\
    \csfont{6} & \tab \csfont{int radius;}              \\
    \csfont{7} & \csfont{$\}$ * node;}                  \\
    \hline
  \end{tabular}
  \biigs
  This node data structure is representative of a circular figure that will be
  drawn on the canvas before it is printed to the output file.
  \smalls
  These attributes have the following meanings:
  \begin{itemize}[label=$\nabla$]
    \item {\csfont{name}} - The name of the node (i.e. what is printed centered
      in the circular node). Via libFreeType we are able to use any font for
      printing this name.
    \item {\csfont{color}} - A pixel representative of the color that the node
      will be drawn onto the canvas with.
    \item {\csfont{fx}} - The column value of the center of the circle (name
      coming from the focal point of a circle which is the center).
    \item {\csfont{fy}} - The row value of the center of the circle (name coming
      from the focal point of a circle which is the center). It should be noted
      that because of the definition of the canvas, a greater value here will
      place the circle lower in the canvas.
    \item {\csfont{radius}} - The radius of the circle representative of the
      node.
  \end{itemize}

\consection{Drawing Nodes and Edges Efficiently}
  For this task I chose Bresenham's Circle drawing algorithm for its runtime in
  the class of $O(1)$ algorithms. Additionally I chose Bresenham's Line drawing
  algorithm for drawing the edges with a time complexity of $O(dx)$ where $dx$
  is the difference between the x coordinates of the endpoints of the line being
  drawn.

\consection{Drawing The Node Collection Legibly}
  One such idea that crossed my mind for this purpose is to draw the nodes
  circularly such that the end user can see the edges from one node to another
  and because in the \eren model of random graphs relative position does not
  matter.
  \smalls
  To accomplish this we must first find the point about which this larger circle
  will be centered (For this program's purpose:
  $$fx_{\text{larger circle}} = \frac{\text{Width}_{\text{canvas}}}{2}$$
  $$fy_{\text{larger circle}} = \frac{\text{Height}_{\text{canvas}}}{2}$$
  Additionally, we must find the radius of this larger circle:
  $$\text{Radius}_{\text{larger circle}}
  = \frac{\text{Height}_{\text{canvas}}}{2}$$
  Next we find the angle between each node:
  $$\theta = \frac{2 \cdot \pi}{\text{qty nodes}}$$
  Finally to find the central point of each node to be drawn (assuming that they
  are linearly indexed):
  $$fx_{\text{node}} = fx_{\text{larger circle}} + \text{Radius}_{\text{larger
  circle}} \cdot \cos{(\theta \cdot \text{node index})}$$
  $$fy_{\text{node}} = fx_{\text{larger circle}} + \text{Radius}_{\text{larger
  circle}} \cdot \sin{(\theta \cdot \text{node index})}$$
  Finally we draw each node.

\consection{Outputing Images}
  Using these data structures my program can output one of two file formats:
  \begin{itemize}[label=$\nabla$]
    \item {\csfont{NetPBM}} (Specifically {\csfont{.ppm}} files).
    \item PNG.
  \end{itemize}

\end{document}
